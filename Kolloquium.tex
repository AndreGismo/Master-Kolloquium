%%%%%%%%%%%%%%%%%%%%%%%%%%%%%%%%%%%%%%%%%%%%%%%%%%%%%%%%%%%%%%%%%%%%%%%%%%%%%%%
% Präsi für Kolloquium
%%%%%%%%%%%%%%%%%%%%%%%%%%%%%%%%%%%%%%%%%%%%%%%%%%%%%%%%%%%%%%%%%%%%%%%%%%%%%%%

\documentclass[aspectratio=169]{beamer}

\usepackage[ngerman]{babel}
\usepackage{fontspec}
\usepackage{microtype}
\usepackage{selnolig}
\usepackage{csquotes}
\usepackage{amsmath, amssymb}
\usepackage{bm}
\usepackage{soul}
\usepackage{booktabs}
\usepackage{mwe}
\usepackage[style=ieee, backend=biber]{biblatex}
\usepackage{tikz}
\usetikzlibrary{
	positioning,
	calc,
	decorations.shapes,
	decorations.pathmorphing,
	decorations.pathreplacing,
}
\usepackage{listings}


\graphicspath{{img/}}


\addbibresource{refs.bib}


\hypersetup{
	pdfauthor={André Ulrich},
	pdftitle={Kolloquium zur Masterarbeit},
}


\lstset{
	basicstyle=\ttfamily,
	morekeywords={
		{\verb|\begin|},
		\textbackslash end,
		\textbackslash documentclass,
	},
	keywordstyle=\color{blue},
	frame=single,
}


\newcommand*{\param}[1]{$\langle$\textit{\ttfamily #1}$\rangle$}
\newcommand*{\tb}{\textbackslash}

\makeatletter
    \newenvironment{withoutheadline}{
        \setbeamertemplate{headline}[default]
        \def\beamer@entrycode{\vspace*{-\headheight}}
    }{}
\makeatother


\pgfdeclarelayer{background}
\pgfsetlayers{background, main}


\usetheme{frankfurt}
\setbeamertemplate{footline}[frame number]
\setbeamertemplate{navigation symbols}{}


\begin{document}
\author{André Ulrich}
\title{Algorithmus zur optimierten Ladung von Elektrofahrzeugen in einem
gemeinsamen Netzzweig}
\subtitle{Master Kolloquium}
\date{
	\today\bigskip
	
	\begin{tabular}{@{}ll@{}}
	Betreuer: & Prof.\,Dr. Eberhard Waffenschmidt \\
	Ko-Referent: & Prof.\,Dr. Ingo Stadler \\
	\end{tabular}
}
\titlegraphic{
	\pgfputat{\pgfxy(6,0.4)}{
		\includegraphics[scale=0.05]{TH_Koeln_Logo.pdf}
	}
}


\setbeamertemplate{footline}[default]
\begin{withoutheadline}
\begin{frame}[noframenumbering]
	\maketitle
\end{frame}
\end{withoutheadline}


%\setbeamertemplate{footline}[frame number]
\begin{withoutheadline}
\begin{frame}[noframenumbering]{Inhalt}
	\tableofcontents
\end{frame}
\end{withoutheadline}


\section{Hintergrund und Aufgabenstellung}
\setbeamertemplate{footline}[frame number]

\begin{frame}{Hintergrund}
	\begin{itemize}[<+->]
		\item PROGRESSUS-Projekt: weniger Netzüberlastung, mehr BEVs laden
		\cite{noauthor_progressus_nodate}
		\item Entwicklung von Optimierungsalgorithmen zur koordinierten Ladung 
		von BEVs
		\item Bestehende Lösungsansätze:
		\begin{itemize}
			\item GridTopologyEstimation (GTE) zum Bestimmen der Netztopologie
			\item GridStateEstimation (GSE) zum Bestimmen des Netzzustands
		\end{itemize}
		\item \so{ABER:} Bisher nur eingeschränkte Ansätze zur Optimierung
		wie \cite{iserloh2021} oder \cite{koese2022} vorhanden
	\end{itemize}
\end{frame}


\begin{frame}{Hintergrund}
	\begin{figure}
		\centering
		%%%%%%%%%%%%%%%%%%%%%%%%%%%%%%%%%%%%%%%%%%%%%%%%%%%%%%%%%%%%%%%%%%%%%%%%%%%%%%%
% Bild zum Überblick der aktuellen Aufgaben im PROGRESSUS-Projekt
%%%%%%%%%%%%%%%%%%%%%%%%%%%%%%%%%%%%%%%%%%%%%%%%%%%%%%%%%%%%%%%%%%%%%%%%%%%%%%%

\begin{tikzpicture}[
	every node/.style={
		font=\footnotesize
	},
	block/.style={
		draw,
		rectangle,
		align=center,
		minimum width=1cm,
		execute at begin node=\strut
	}
]

	\node[block, rounded corners=3pt](SMG){\strut SMG};
	
	\node[block, right=2.5 of SMG](GTE){GTE};
	
	\node[block, above=2 of SMG, rounded corners=3pt](hardware){Hardware};
	
	\node[block, above=2 of GTE](software){Software};
	
	\node[block, right=1.5 of GTE](GSE){GSE};
	
	\node[block, fill=orange!50,  right=1.5 of GSE , minimum height=3.5cm,
	minimum width=1.5cm](GLO){\strut GLO};
	
	\node[block, right=1.5cm of GLO, minimum height=2.5cm,
	minimum width=1.5cm](EMO){EMO};
	
	\node[block, above=0.8cm of EMO, fill opacity=0, draw opacity=0](GLS){GLS};
	
	
	% Pfeil von SMG zu GSE
	\draw[-stealth] (SMG.270) -- +(0,-0.4) -| (GSE.240) node[pos=0.25, below]
	{$I$, $U$, $\cos\varphi $};
	
	\draw[-stealth] (SMG.east) -- (GTE.west) node[pos=0.5, below]
	{$I$, $U$, $\cos\varphi $};
	
	\draw[-stealth] (GTE.east) -- (GSE.west) node[pos=0.5, below]
	{$\bm{Y}$};
	
	\draw[dashed, -stealth] (GTE.north) |- (GLO.135) node[pos=0.75, above]
	{$\bm{Y}$};
	
	\draw[dashed, -stealth] (GSE.east) -- (GLO.west) node[pos=0.5, below]
	{$I^\mathsf{HH}$};
	
	\draw[dashed, -stealth] (GLO.225) -| (GSE.300)
	node[pos=0.25, below]{$I^\mathsf{BEV}$};
	
%	\draw[-stealth] (EMO.135) -- (GLO.45) node[pos=0.5, above]
%	{$U^\mathrm{Knoten}$};

	\draw[-stealth] (EMO.135) -- (EMO.135 -| GLO.east)
	node[pos=0.5, above]{$U^\mathsf{Knoten}$};
	
	\draw[-stealth] (GLO.315) -- (EMO.225) node[pos=0.5, above]
	{$I^\mathsf{BEV}$};
	
	\draw[-stealth] (GLS.west) -| (GLO.north) node[pos=0.25, above]
	{Szenarien};
	
	\begin{pgfonlayer}{background}
		\path(GLO.south west)+(-0.2,-0.8) node (helper-a){};
		\path(GLO.north -| EMO.east)+(0.2,1.3) node (helper-b){};
		\path[fill=yellow!20, rounded corners, draw=black!50, dashed]
		(helper-a) rectangle (helper-b);
	\end{pgfonlayer}
	
	\node at ($($(helper-a) !0.5! (helper-b)$) - (0,2.5)$){Teil dieser Arbeit};

\end{tikzpicture}
	\end{figure}
\end{frame}


\begin{frame}{Aufgabenstellung}
	\begin{enumerate}[<+->]
		\item Entwicklung eines Optimierungsalgorithmus zur optimalen 
		Koordination der Ladevorgänge von BEVs
		\item Verknüpfung von Optimierung und Simulationsumgebung
		\item Ermöglichen eines parallelen Betriebs von Optimierung und
		Simulationsumgebung
	\end{enumerate}
\end{frame}


\begin{frame}{Aufgabenstellung}
	Der Optimierungsalgorithmus soll:
	\begin{itemize}[<+(1)->]
		\item Zunächst nur einzelne Netzstrahlen betrachten
		\item Netzauslastung maximieren, um möglichst viele BEVs zu laden
		\item Gleichzeitig jedoch Netzüberlastung verhindern
		\item Nach Möglichkeit Diskriminierung beim Laden ausschließen
		\item Automatisch skalieren mit dem betrachteten Netzstrahl
		\item Auch auf einem Raspberry Pi in \enquote{vertretbarer} Zeit 
		ausführbar sein
	\end{itemize}
	
	\onslide<8>{
		\begin{block}{also}
			Einschränkungen bisheriger Arbeiten überwinden und 
			\enquote{Gratwanderung vollführen}.
		\end{block}
	}
\end{frame}

\section{Theorie}

\begin{frame}{Prinzip des Optimierungsalgorithmus}
	\begin{figure}
		\centering
		%%%%%%%%%%%%%%%%%%%%%%%%%%%%%%%%%%%%%%%%%%%%%%%%%%%%%%%%%%%%%%%%%%%%%%%%%%%%%%%
% Bild für die Input-/Output-Parameter
%%%%%%%%%%%%%%%%%%%%%%%%%%%%%%%%%%%%%%%%%%%%%%%%%%%%%%%%%%%%%%%%%%%%%%%%%%%%%%%

\begin{tikzpicture}[
		block/.style={
			font=\tiny,
			draw,
			rectangle,
			minimum width={3.5cm}
			}
		]
		\node[block](block0){\strut Haushaltslastprofile};
		
		\node[block,
			below=0.3 of block0](block1){\strut Transformatorleistung};		
		
		\node[block,
			below=0.2 of block1](block2){\strut Anzahl der Knoten};
			
		\node[block,
			below=0.2 of block2](block3){\strut Anzahl Ladestationen};
			
		\node[block,
			below=0.2 of block3](block4){\strut Leitungsimpedanzen};
			
		\node(gridtopology)[below=0.1 of block4]{\tiny\strut 
		Netztopologie};
			
		\node[block,
			below=0.3 of gridtopology](block5){\strut Start/Ziel SOC};
			
		\node[block,
			below=0.2 of block5](block6){\strut Start/Ziel Zeit};
			
		\node(customer)[below=0.1 of block6]{\tiny\strut Kundenwünsche};
			
		\node[block,
			below=0.3 of customer](block7){\strut Zeitliche Auflösung};
			
		\node[block,
			below=0.2 of block7](block8){\strut Weite des Horizonts};
			
		\node(timespecs)[below=0.1 of block8]{\tiny\strut Zeitliche
		Betrachtung};
			
		\node[block,
			right=2 of gridtopology,
			minimum height=5cm](GLO){GridLineOptimizer (GLO)};
			
		\node[block,
			right=of GLO, align=left](res){Zeitreihen für \\ Ladestrom und 
			SOC};
			
		
		\draw[-stealth](block0.east) -- (GLO);
		\draw[-stealth](block1.east) -- (GLO);
		\draw[-stealth](block2.east) -- (GLO);
		\draw[-stealth](block3.east) -- (GLO);
		\draw[-stealth](block4.east) -- (GLO);
		\draw[-stealth](block5.east) -- (GLO);
		\draw[-stealth](block6.east) -- (GLO);
		\draw[-stealth](block7.east) -- (GLO);
		\draw[-stealth](block8.east) -- (GLO);
		\draw[-stealth](GLO) -- (res);
		
		
		\begin{pgfonlayer}{background}
			\path(block1.west)+(-0.1, 0.4) node (a) {};
			\path(block4.east)+(0.1, -0.9) node (b) {};
			\path[fill=yellow!20, rounded corners, draw=black!50, dashed]
			(a) rectangle (b);
			
			\path(block5.west)+(-0.1, 0.4) node (a) {};
			\path(block6.east)+(0.1, -0.9) node (b) {};
			\path[fill=red!20, rounded corners, draw=black!50, dashed]
			(a) rectangle (b);
			
			\path(block7.west)+(-0.1, 0.4) node (a) {};
			\path(block8.east)+(0.1, -0.9) node (b) {};
			\path[fill=green!20, rounded corners, draw=black!50, dashed]
			(a) rectangle (b);
		\end{pgfonlayer}
	\end{tikzpicture}
	\end{figure}
\end{frame}


\begin{frame}{Optimierung mit Mathematical Programming}
	\begin{columns}
		\begin{column}{.6\textwidth}
			\begin{itemize}
				\item Mathematical Programming als strukturierte 
				Darstellungsweise eines Optimierungsproblems
				\item Beschreibung mithilfe von Zielfunktion und Restriktionen
				\item Formulierug als Linear Programmig
			\end{itemize}
		\end{column}
		\begin{column}{.4\textwidth}
			\begin{align*}
				\begin{split}
		\text{max.}\quad&\sum_{i\in\mathcal{I}}c_i\cdot x_i \\
		\text{s.\,t.}\quad&\sum_{i\in\mathcal{I}}a_{j,i}\cdot x_i\leq b_j\quad
		\forall j\in\mathcal{J} \\
		& x_i\in\mathbb{R}_0^+\quad\forall i\in\mathcal{I}
				\end{split}
			\end{align*}
		\end{column}
	\end{columns}
\end{frame}


\begin{frame}{Formulierung als Linear Programming}
	\begin{description}[<+->]
		\item[Zielfunktion] Maximieren der der Ladeströme an allen Knoten zu 
		jeweils allen Zeitpunkten
		\item[Restriktion] Die Spannung am letzten Knoten darf nicht aus dem
		Spannungsband fallen
		\item[Restriktion] Der Transformator darf nicht überlastet werden
		\item[Restriktion] Keine der Leitungen darf überlastet werden
		\item[Restriktion] Energieerhaltung beim Laden
	\end{description}
	\onslide<6>{
		\begin{block}{$\dots$ und außerdem:}
			4 Weitere Restriktionen für \enquote{faires} Laden
		\end{block}
	}
\end{frame}


\begin{frame}{Rolling Horizon für große Optimierungsprobleme}
	\begin{itemize}[<+->]
		\item Großer Zeithorizont, zu dem ein großes Optimierungsproblem
		korrespondiert
		\item Großen Zeithorizont zerlegen in mehrere kleine Horizonte
		\item Die zu den einzelnen Horizonten korrespondierenden 
		Optimierungsprobleme lösen
		\item Die Gesamtlösung ergibt sich aus den einzelnen Teillösungen
	\end{itemize}
\end{frame}


\begin{frame}{Rolling Horizon für große Optimierungsprobleme}
	\begin{figure}
		\centering
		%%%%%%%%%%%%%%%%%%%%%%%%%%%%%%%%%%%%%%%%%%%%%%%%%%%%%%%%%%%%%%%%%%%%%%%%%%%%%%%
% Schematische Darstellung eines Rolling Horizon
%%%%%%%%%%%%%%%%%%%%%%%%%%%%%%%%%%%%%%%%%%%%%%%%%%%%%%%%%%%%%%%%%%%%%%%%%%%%%%%

\begin{tikzpicture}
	\draw[->] (0,0) -- (10,0) node[right]{$t$};
	% 1. Horizont
	\node[left] at (0,-0.5) {$\mathcal{T}_1$};
	\foreach \x in {0,1,...,5}{
		\draw(\x,-.75) rectangle (\x+1,-.25);
		\fill[black, opacity={0.6-\x/10}](\x,-.75) rectangle (\x+1,-.25);
	}
	\draw[-stealth] (1.5,-0.5) .. controls (1.8,-.75) and (1.8,-1)
	.. (1.5,-1.25);
		
	% 2. Horizont
	\node[left] at (0,-1.25) {$\mathcal{T}_2$};
	\foreach \x in {1,2,...,6}{
		\draw(\x,-1.5) rectangle (\x+1,-1);
		\fill[black, opacity={0.7-\x/10}](\x,-1.5) rectangle (\x+1,-1);
	}
	\draw[-stealth] (2.5,-1.25) .. controls (2.8,-1.5) and (2.8,-1.75) 
	.. (2.5,-2);
		
	% 3. Horizont
	\node[left] at (0,-2) {$\mathcal{T}_3$};
	\foreach \x in {2,3,...,7}{
		\draw (\x,-2.25) rectangle (\x+1,-1.75);
		\fill[black, opacity={0.8-\x/10}](\x,-2.25) rectangle (\x+1,-1.75);
	}
	\draw[-stealth] (3.5,-2) .. controls (3.8,-2.25) and (3.8,-2.5) 
	.. (3.5,-2.75);
	
	% 4. Horizont
	\node[left] at (0,-2.75) {$\mathcal{T}_4$};
	\foreach \x in {3,4,...,8}{
		\draw (\x,-3) rectangle (\x+1,-2.5);
		\fill[black, opacity={0.9-\x/10}](\x,-3) rectangle (\x+1,-2.5);
	}
	
	% oben geschweifte Klammer, welche die Weite eines Horizonts angibt
	\draw[decorate, decoration=brace] (0, 0.3) -- (6, 0.3)
	node[pos=0.5, above=1mm]{$w$};
		
	% unten das Ergebnis zeichnen
	\node[left] at (0,-4) {$\mathcal{T}$};
	\foreach \x in {0,1,...,9}{
		\draw (\x,-3.75) rectangle (\x+1,-4.25);
	}
	
	\foreach \x in {0,1,...,3}{
		\fill[black, opacity={0.6}] (\x,-3.75) rectangle (\x+1,-4.25);
	}
	
	% Pfeile, die anzeigen, dass die jeweils ersten Ergebnisse von
	% den Horizonten übernommen werden
	\draw[-stealth] (0.5,-0.9) -- (0.5,-3.6);
	\draw[-stealth] (1.5,-1.65) -- (1.5,-3.6);
	\draw[-stealth] (2.5,-2.4) -- (2.5,-3.6);
	\draw[-stealth] (3.5,-3.15) -- (3.5,-3.6);
	
	% diagonale Pünktchen, die andeuten, dass noch mehr Horizonte folgen
	\node at (6.5,-3.2){$\ddots$};
	\node[left] at (-0.2,-3.1){$\vdots$};
	
\end{tikzpicture}
	\end{figure}
\end{frame}


\section{Praktische Umsetzung}

\begin{frame}{Optimierungsprobleme mit Python lösen}
	\begin{itemize}
		\item Python als Programmiersprache
		\item Pyomo als Optimierungsframework
		\item glpk als Solver
	\end{itemize}
\end{frame}


\begin{frame}{Aufbau der Software zur Optimierung}
	Mehrere Klassen:
	\begin{itemize}[<+(1)->]
		\item Eigene Klasse \texttt{GridLineOptimizer} für Netzstrahl und 
		Logik der Optimierung
		\item Eigene Klasse \texttt{BatteryElectricVehicle} für BEVs
		\item Eigene Klasse \texttt{Household} für Haushaltslasten
	\end{itemize}
	
	\onslide<5->{
		\begin{block}{Modularer Aufbau}
			Bessere Wartbarkeit und Portabilität
		\end{block}
	}
\end{frame}


\begin{frame}{Funktionsprinzip der Software zur Optimierung}
	\begin{figure}
		\centering
		%%%%%%%%%%%%%%%%%%%%%%%%%%%%%%%%%%%%%%%%%%%%%%%%%%%%%%%%%%%%%%%%%%%%%%%%%%%%%%%
% flow chart für das Funktionsprinzip der Software
%%%%%%%%%%%%%%%%%%%%%%%%%%%%%%%%%%%%%%%%%%%%%%%%%%%%%%%%%%%%%%%%%%%%%%%%%%%%%%%

\begin{tikzpicture}[
	block/.style={
		font=\tiny,
		draw,
		rectangle,
		minimum width={4cm},
		execute at begin node=\strut,
		}
	]
	
	\node[block](grid){Netztopologie};
	
	\node[block,
		right=0.3 of grid](time){Zeitliche Betrachtung};
		
	\node[block,
		below left=0.5 and 0.15 of time.south](sets){Entsprechende Sets 
		erzeugen};
		
	\node[block,
		below=0.3 of sets](params){Parameter erzeugen};
		
	\node[block,
		below=0.3 of params](vars){Variablen erzeugen};
		
	\node[block,
		below=0.3 of vars](bounds){Ober-/Untergrenzen festlegen};
		
	\node[block,
		below=0.3 of bounds](target){Zielfunktion erzeugen};
		
	\node[block,
	 	below=0.3 of target](constraints){Restriktionen erzeugen};
	 	
	 \node[block,
	 	right=0.9 of params](households){Haushaltslastprofile};
	 	
	 \node[block,
	 	right=0.9 of bounds](cust-wish){Kundenwunsch};
	 	
	 \node[block,
	 	below=0.5 of constraints](result){Optimierungsmodel lösen};
	 	
	 \node[left=0.3 of sets,
	 	rotate=90](model){\scriptsize Optimierungsmodel erzeugen};
	 	
	 \node[above=0.1 of cust-wish](BEV){\footnotesize\ttfamily 
	 BatteryElectricVehicle};
	 
	 \node[above=0.1 of households](HH){\footnotesize\ttfamily Household};
	 
	 \node[left=1 of params,
	 	rotate=90](GLO){\footnotesize\ttfamily GridLineOptimizer};
	 	
	 	
	 \draw[-stealth](grid)--(sets);
	 \draw[-stealth](time)--(sets);
	 \draw[-stealth](sets)--(params);
	 \draw[-stealth](params)--(vars);
	 \draw[-stealth](vars)--(bounds);
	 \draw[-stealth](bounds)--(target);
	 \draw[-stealth](target)--(constraints);
	 \draw[-stealth](constraints)--(result);
	 \draw[-stealth](cust-wish)--(bounds);
	 \draw[-stealth](households)--(params);
	 
	 \begin{pgfonlayer}{background}
	 	\path(sets.west)+(-0.6, 0.5) node (a) {};
		\path(constraints.east)+(0.1, -0.5) node (b) {};
		\path[fill=yellow!20, rounded corners, draw=black!50, dashed]
		(a) rectangle (b);
	 \end{pgfonlayer}

\end{tikzpicture}
	\end{figure}
\end{frame}


\section{Ergebnisse}

\begin{frame}{Untersuchung anhand von Szenarien}
	\begin{itemize}[<+->]
		\item Definieren unterschiedlicher Szenarien
		\begin{itemize}
			\item Transformatorleistung
			\item Anzahl Knoten
			\item Position der Ladesäulen
			\item $\dots$
		\end{itemize}
		\item Optimierung ausführen
		\item Ergebnisse mithilfe der Simulationsumgebung validieren
	\end{itemize}
\end{frame}


\section{Fazit}

\begin{frame}{Fazit}
	bla bla
\end{frame}


\begin{frame}{Literatur}
	\printbibliography
\end{frame}


\begin{frame}{Formulierung als Linear Programming}
	Die Zielfunktion: Maximieren der Ladeströme zu allen Zeitpunkten
	\begin{equation*}
		\text{max.}\quad\sum_{t\in\mathcal{T}}\bigg(\sum_{n\in
		\mathcal{N_\mathsf{BEV}}}\!\!I^\mathsf{BEV}_{t,n}\bigg)
	\end{equation*}
\end{frame}

\end{document}